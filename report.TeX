%!TEX TS-program = lualatex
%!TEX encoding = UTF-8 Unicode
\title{Project Report}
%
%%% Preamble
\documentclass[a4paper,final]{scrartcl}
%\documentclass[paper=a4, fontsize=14pt]{scrartcl}
\usepackage[english]{babel}
\usepackage{polyglossia}
%usepackage[T1]{fontenc}
\usepackage{kpfonts, Baskervaldx}
\usepackage{bold-extra}
\usepackage{tikz}
\usetikzlibrary{automata, positioning, arrows}
\usepackage{fontspec}
\usepackage{titlesec}
\newfontfamily\headingfont[]{baskervaldx}
	\titleformat{\section}{\centering\scshape\headingfont}{\thesection}{1em}{\LARGE}
	\titleformat{\subsection}{\centering\scshape\headingfont}{\thesubsection}{1em}{\large}

%\setmainfont{baskervaldx}[Numbers=OldStyle]
\setmonofont{JetBrains Mono}
%\setmathsfont{kpfonts}

\setmainlanguage[]{english}
\usepackage[protrusion=true, expansion=true]{microtype}	
\usepackage{amsmath,amsfonts,amsthm} % Math packages
%\usepackage[luatex]{graphicx}	
\usepackage{url}
\usepackage{float}
\usepackage{listings}
%\usepackage[standardsections]{scrhack}

%%% Custom headers/footers (fancyhdr package)
%%% Custom headers/footers (fancyhdr package)
\usepackage{scrlayer-fancyhdr}
\pagestyle{fancyplain}
\fancyhead{}											% No page header
\fancyfoot[L]{}											% Empty 
\fancyfoot[C]{}											% Empty
\fancyfoot[R]{\thepage}									% Pagenumbering
\renewcommand{\headrulewidth}{0pt}			% Remove header underlines
\renewcommand{\footrulewidth}{0pt}				% Remove footer underlines
\setlength{\headheight}{13.6pt}



%%% Equation and float numbering
\numberwithin{equation}{section}		% Equationnumbering: section.eq#
\numberwithin{figure}{section}			% Figurenumbering: section.fig#
\numberwithin{table}{section}				% Tablenumbering: section.tab#

%% FSM 
\tikzset
{
	->, % makes the edges directed
	node distance=2.5cm, % Minimum distance between two nodes. Change if necessary.
	every state/.style={semithick, fill=gray!10},
	initial text={}, % No label on start arrow
	double distance=2pt% Adjust appearance of accept states
}
\let\epsilon\varepsilon

%%% Maketitle metadata
\newcommand{\horrule}[1]{\rule{\linewidth}{#1}} 	% Horizontal rule

\title{
		%\vspace{-1in} 	
		\usefont{OT1}{bch}{b}{n}
		\normalfont \normalsize \textsc{Engineering Software 3} \\ [15pt]
		\horrule{0.5pt} \\[0.4cm]
		\huge User guide - Traffic lights \\
		\horrule{2pt}\\[0.5cm]
}
\author{
		\normalfont\normalsize
        Shaolun Wang\\[-3pt]\normalsize
        \today
}
\date{}


%%% Begin document
\begin{document}
\maketitle
\section{Application Design}
\subsection{Description}
The Traffic light designed here, would work in such way:
	\begin{enumerate}
		\item Initial traffic state:\\ All the traffic lights are at red. Pedestrian traffic lights are not shown, and LED not lighted either.
		\item As the time goes, each traffic light would be shown in the order of: $$\ldots red \rightarrow yellow \rightarrow green \rightarrow yellow \rightarrow red \ldots$$
		After each of this sequence being executed, we move on from one traffic light to another, in the order of $$\ldots tr_1 \rightarrow tr_2 \rightarrow tr_3 \ldots$$
		It will be repeated over and over again until a pedestrian button is pressed. 

		\item If a pedestrian button is pressed:
			\begin{enumerate}
				\item  we first finish the current traffic light display, while setting the pedestrian LED as on (as a wait signal). Then we start to display the pedestrian light as green for 5 seconds. At the last two second of the crossing time, the light will blink, indicating it will end soon.
				\item When the pedestrian light turns red, the traffic light would continue to display, following the current cycle.

		\end{enumerate}
		
	\end{enumerate}
\subsection{Specifications}
The board has the following inputs and outputs:
\begin{enumerate}
	\item Inputs:
		\begin{enumerate}
			\item BTNR, for pedestrian crossing.
			\item BTNC, for resetting the program.
		\end{enumerate}
	\item Outputs:
		\begin{enumerate}
			\item LED, indicating the pedestrian should wait until the light turns red on the VGA display.
			\item VGA display, consists of 9 regions, Each Column on the left side represents a traffic light. Region 10 represents the pedestrian light. Region 10 will blink once the pedestrian crossing timer hits 2s.
			Regions as shown:

			\begin{figure}[H]
				\centering
				\includegraphics[width=0.6\textwidth]{region.png}
				\caption{Region used in the design}
				\label{Region}
			\end{figure}				

			\item Seg7 display, showing the timer of current crossing (car / pedestrian).  
			\begin{itemize}
				\item  Time given for each traffic light to be displayed: 4s.
				\item Time given for pedestrian light to be displayed: 5s. 
			\end{itemize}
		\end{enumerate}
\end{enumerate}
	
%\begin{align} 
%	\begin{split}
%	(x+y)^3 	&= (x+y)^2(x+y)\\
%					&=(x^2+2xy+y^2)(x+y)\\
%					&=(x^3+2x^2y+xy^2) + (x^2y+2xy^2+y^3)\\
%					&=x^3+3x^2y+3xy^2+y^3
%	\end{split}					
%\end{align}

\section{Functions and Implementations}
\subsection{Functions}
\subsection{Main File}
	\begin{enumerate}
		\item \texttt{$vga\_init()$}
				This function sets up the VGA initial states. Red for region 0,3,6,10, white for other regions.
		
		\item \texttt{$pedestrian\_check$} \hfill \\
				This function checks whether a pedestrian button is being pressed or not. If pressed, an interrupt will happen after the sequence of current traffic light is finished.
		\item $tr_i\_switch$ and $FSM$ \hfill \\
				These three functions are FSMs that take control of each traffic light. It has 5 states: $R, (R)Y, G, Y, R$. 
				The ending state is the same as the starting state,
				which means that we could design it with a machine consists of 4 states.
				\begin{figure}[H] 
					\centering % centers the figure
					\begin{tikzpicture}
						\node[state, initial]    (r1) {$R_1$};
						\node[state, right of=r1] (ry1) {$RY_1$};
						\node[state, below right of=ry1] (y1) {$Y_1$};
						\node[state, right of=y1] (g1) {$G_1$};

						\node[state, below of=r1] (r2) {$R_2$};
						\node[state, right of=r2] (ry2) {$RY_2$};
						\node[state, below right of=ry2] (y2) {$Y_2$};
						\node[state, right of=y2] (g2) {$G_2$};

						\node[state, below of=r2] (r3) {$R_3$};
						\node[state, right of=r3] (ry3) {$RY_3$};
						\node[state, below right of=ry3] (y3) {$Y_3$};
						\node[state, right of=y3] (g3) {$G_3$};
						\draw 
						(r1) edge[above] node{$l_1 = 1$} (ry1)
						(ry1) edge[bend left, above] node{$1_1 = 2$} (g1)
						(g1) edge[above] node{$l_1 = 3$} (y1)
						(y1) edge[below] node{$l_1 = 0$} (r1)

						(r1) edge[right] node{$TR_2$} (r2)

						(r2) edge[above] node{$l_2 = 1$} (ry2)
						(ry2) edge[bend left, right] node{$1_2 = 2$} (g2)
						(g2) edge[above] node{$l_2 = 3$} (y2)
						(y2) edge[below] node{$l_2 = 0$} (r2)
					
						(r2) edge[right] node{$TR_3$} (r3)

						(r3) edge[above] node{$l_3 = 1$} (ry3)
						(ry3) edge[bend left, right] node{$1_3 = 2$} (g3)
						(g3) edge[above] node{$l_3 = 3$} (y3)
						(y3) edge[below] node{$l_3 = 0$} (r3)

						(r3) edge[bend left, left] node{$TR_1$} (r1);
					\end{tikzpicture}
					\caption{FSM of the Traffic Light Control}
					\label{fig:fsm}
				\end{figure}
				In this diagram, $l_i$ is controlling the display of each traffic light. It is incremented in the expression of \texttt{++$l_i$ \% $4$}. This would correctly set the states back to red, which is the initial state. And at the initial state, if $TR_{i}$ is triggered, it will start lighting up the next set of traffic lights.

				The FSM is implemented with two parts, one is $tr_1\_switch()$ functions: for each of the three traffic light, the other one is the general update rule of which set of traffic lights would be displayed; The other one is $FSM()$, this will control all the values of $TR_i$.
		\item $pd\_LED$ control \hfill \\
				This function is to correctly time the LED with the VGA display. When the pedestrian button is pressed, if it's not yet red light for the traffics, then the LED would turn on. Once the region 10 (VGA for pedestrian light) turns green, the LED would go off.
		\item $hwtimerISR$ \hfill \\
				This is the interrupt function. The interrupt function would check whether it's the turn for the pedestrians to go. If so, it would turn region 10 to green and do blink until boolean value $go$ is set to false. And if pedestrian cannot go, it would keep executing FSM with given states.

				It will keep calling LED, but LED might not display due to no LED is assigned.

				It will also keep calling $displayDigit()$ as the FSM runs. This is a separate interrupt call, but it can work "in parallel" to the FSM and pedestrian lights \& LED.
		\item $seg7\_display$ module \hfill \\
				This module controls the seg7 display. By calling $displayNumber(number)$, it will display number in between [0, 9999]. $DisplayNumber$ would call $displayDigit$ to display each individual digit with less or equal to 4 interrupts, where $displayDigit$ is being set up in $hwtimerISR$ function.
	\end{enumerate}
\end{document}